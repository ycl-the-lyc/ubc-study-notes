%% Standard start of a latex document
\documentclass[letterpaper,12pt]{article}
%% Always use 12pt - it is much easier to read

%% Set some names and numbers here so we can use them below
\newcommand{\myname}{Ashok Kumar} %%%%%%%%%%%%%%% ---------> Change this to your name
\newcommand{\mynumber}{0.5772156649} %%%%%%%%%%%%%%% ---------> Change this to your student number
\newcommand{\hw}{0} %%%%%%%%%%%%%%% --------->  set this to the homework number

%%%%%%
%% There is a bit of stuff below which you should not have to change
%%%%%%

%% AMS mathematics packages - they contain many useful fonts and symbols.
\usepackage{amsmath, amsfonts, amssymb}

%% The geometry package changes the margins to use more of the page, I suggest
%% using it because standard latex margins are chosen for articles and letters,
%% not homework.
\usepackage[paper=letterpaper,left=25mm,right=25mm,top=3cm,bottom=25mm]{geometry}
%% For details of how this package work, google the ``latex geometry documentation''.

%%
%% Fancy headers and footers - make the document look nice
\usepackage{fancyhdr} %% for details on how this work, search-engine ``fancyhdr documentation''
\pagestyle{fancy}
%%
%% The header
\lhead{Mathematics 220} % course name as top-left
\chead{Homework \hw} % homework number in top-centre
\rhead{ \myname \\ \mynumber }
%% This is a little more complicated because we have used `` \\ '' to force a line-break between the name and number.
%%
%% The footer
\lfoot{\myname} % name on bottom-left
\cfoot{Page \thepage} % page in middle
\rfoot{\mynumber} % student number on bottom-right
%%
%% These put horizontal lines between the main text and header and footer.
\renewcommand{\headrulewidth}{0.4pt}
\renewcommand{\footrulewidth}{0.4pt}
%%%

%%%%%%
%% We shouldnt have to change the stuff above, but if you want to add some newcommands and things like that, then putting them between here and the ``\begin{document}'' is a good idea.
%%%%%%


\begin{document}
Homework 0 does not contain any mathematics --- it is just for you to practice using latex. All I want you to do is to try to reproduce this document as well as you can. You do not have to hand it in. I don't mind if you work in small groups, but just copying it directly from a friend isn't going to help you later in the term.


\textbf{Questions:}

%%
%% There are 2 list environments, itemize and enumerate. They are almost identical, but each item in itemize is started with a bullet or dot, while each item in enumerate is numbered.
%%
\begin{enumerate}
%% This is where your actual homework will go.
 \item Your solution to question 1.

 \item Your solution to question 2.

 \item Your solution to question 3.

 \item And now we are on to question 4.

\item Question 5.

\item Question 6.
\end{enumerate}

%% Anything that comes after the ``\end{document}'' will be ignored.
\end{document}
